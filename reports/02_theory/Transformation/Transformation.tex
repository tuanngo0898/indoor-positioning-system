
\documentclass[12pt]{report}
\usepackage[utf8]{inputenc}
\usepackage{mathtools,amssymb,stackengine}

\begin{document}

\chapter{Transformation}

\section{Matrix transformations}
Objectives:
\begin{enumerate}
    \item Learn to view a matrix geometrically as a function
    \item Learn example of matrix transformations: reflection, dilation, rotation, shear, projection
    \item Understand vocabulary surrounding transformation: domain, codomain, range.
    \item Understand domain, codomain, and range of a matrix transformation.
    \item Pictures: common matrix transformations.
    \item Vocabulary words: transformation/function, domain, codomain, range, identity transformation, matrix transformation.
\end{enumerate}

Objectives:
\begin{enumerate}
    \item Learn how to verify that a transformation is linear, prove that a transformation is not linear.
    \item Understand the relationship between linear transformations and matrix transformations.
    \item Recipe: Compute the matrix of linear transformation.
    \item Theorem: linear transformations and matrix transformations
    \item Notation: the stand of coordinate vector
    \item Vocabulary words: linear transformation, standard matrix, identity matrix
  \end{enumerate}


In mathematics, a linear transformation (also called linear mapping) is a mapping $\mathbb{W} \to \mathbb{V}$ between two modules (for example, two vector spaces) that preserves the operation of addition and scalar multiplication.

\end{document}